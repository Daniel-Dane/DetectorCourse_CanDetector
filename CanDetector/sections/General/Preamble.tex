\documentclass[
	twoside=semi,
	a4paper,
	symmetric,
	justified
]{sections/CustomStyles/tufte-book}

% Missing packages
\usepackage{feynmp-auto} % FeynMF to make Feynman 
\usepackage{siunitx}     % To show non-italic units in math-mode
%\sisetup{number-unit-product = \hspace{0.16667em plus 0.08334em}}
\usepackage{physics}
\usepackage{nameref}
%\usepackage{float}

% Proper square root symbol
\usepackage{letltxmacro}
\makeatletter
\let\oldr@@t\r@@t
\def\r@@t#1#2{%
	\setbox0=\hbox{$\oldr@@t#1{#2\,}$}\dimen0=\ht0
	\advance\dimen0-0.2\ht0
	\setbox2=\hbox{\vrule height\ht0 depth -\dimen0}%
	{\box0\lower0.4pt\box2}}
\LetLtxMacro{\oldsqrt}{\sqrt}
\renewcommand*{\sqrt}[2][\ ]{\oldsqrt[#1]{#2}}
\makeatother

\title[An alternative analysis of semi-leptonic diboson final states in the boosted regime]{An alternative \\ \noindent analysis of the \\ \noindent semi-leptonic \\ \noindent diboson final \\ \noindent states in the \\ \noindent boosted regime}

\author[Daniel Stefaniak Nielsen]{Daniel Stefaniak Nielsen}
\publisher{MSc thesis \\ \noindent Niels Bohr Institute, University of Copenhagen}


%
% Style settings are set in 'tufte-common.def'
%

%%
% For nice reference listings.
%\usepackage[sorting=none]{biblatex}

%%
% Proper typesetting
\usepackage{microtype}

%% 
% Helvetica available as math sans serif font
\usepackage{sansmath} % Enables turning on sans-serif math mode, and using other environments
%\sansmath % Enable sans-serif math for rest of document

%%
% Just some sample text
\usepackage{lipsum}

%%
% For nicely typeset tabular material
\usepackage{booktabs}

%%
% For graphics / images
\usepackage{graphicx}
\setkeys{Gin}{width=\linewidth,totalheight=\textheight,keepaspectratio}
\graphicspath{{graphics/}}

% The fancyvrb package lets us customize the formatting of verbatim
% environments.  We use a slightly smaller font.
\usepackage{fancyvrb}
\fvset{fontsize=\normalsize}


% Prints an epigraph and speaker in sans serif, all-caps type.
\newcommand{\openepigraph}[2]{%
  %\sffamily\fontsize{14}{16}\selectfont
  \begin{fullwidth}
  \sffamily\large
  \begin{doublespace}
  \noindent\allcaps{#1}\\% epigraph
  \noindent\allcaps{#2}% author
  \end{doublespace}
  \end{fullwidth}
}


% Inserts a blank page
\newcommand{\blankpage}{\newpage\hbox{}\thispagestyle{empty}\newpage}

\usepackage{units}

% Typesets the font size, leading, and measure in the form of 10/12x26 pc.
\newcommand{\measure}[3]{#1/#2$\times$\unit[#3]{pc}}

% Generates the index
\usepackage{makeidx}
\makeindex

% Upright greek letters, e.g. $\upphi$
\usepackage{upgreek}	

% Double-stroke numbers
\usepackage{bbm}

% Allows table cells to span multiple rows
\usepackage{multirow}

% Enables dashed table lines
\usepackage{array}
\usepackage{arydshln}

\setlength\dashlinedash{0.2pt}
\setlength\dashlinegap{1.5pt}
\setlength\arrayrulewidth{0.3pt}

% Enable table notes.
\usepackage{threeparttable}

% Enable strike-out
\usepackage{cancel}

% Enable caption formatting (used in listings and for subfigures).
\usepackage{graphicx}
\usepackage[compatibility=false,font=small]{caption}
\usepackage[font=small]{subcaption}

% Enable listings
\usepackage{listings}

\lstdefinestyle{m_C++}{ % \lstset{ 
  belowcaptionskip=1\baselineskip,
  breaklines=true,
%  xleftmargin=\parindent,
  framexleftmargin=0.75em,
  frame=tb,
  framesep=0.75em,
  tabsize=4,
  language=C,
  showstringspaces=false,
  basicstyle=\footnotesize\ttfamily,
  keywordstyle=\bfseries\color{m_FlatUI_2},		%\bfseries\color{green!40!black},
  commentstyle=\itshape\color{m_FlatUI_1},			%\color{green!40!black},		%\color{purple!40!black},
  identifierstyle=\color{m_FlatUI_5},				%\color{m_backcolor},
  stringstyle=\color{m_FlatUI_2!70!black}, 		% \color{m_FlatUI_4},	  		%\color{orange},
  backgroundcolor=\color{m_FlatUI_3},
  numbers=left,
  numberstyle=\scriptsize\color{black!50}\ttfamily,
  numberbychapter=true,
}

\lstdefinelanguage{pseudo}{
	morekeywords={if, then, else, do, while, return, for, each, all, in, otherwise, begin, end, get, remove},
	sensitive=false,
	morecomment=[l]{//},
	morecomment=[s]{/*}{*/},
	morestring=[b]",
}


\lstdefinestyle{m_pseudo}{ % \lstset{ 
  belowcaptionskip=1\baselineskip,
  breaklines=true,
%  xleftmargin=\parindent,
  framexleftmargin=0.75em,
  frame=tb,
  framesep=0.75em,
  tabsize=4,
  language=pseudo,
  showstringspaces=false,
  basicstyle=\small\sffamily,
  keywordstyle=\bfseries,
  commentstyle=\itshape,	
  identifierstyle={},
  stringstyle={},
  backgroundcolor=\color{m_FlatUI_3},
  numbers=left,
  numberstyle=\scriptsize\color{black!50}\sffamily,
  numberbychapter=true,
  flexiblecolumns=true,
}

\lstset{style=m_C++}

\captionsetup[lstlisting]{ singlelinecheck=false, margin=0pt, font={footnotesize} }


% Custom commands.
% ---------------------------------------------------------------------------------------------------------------------------------------------------------

% * Manchet: Styles the first paragraph in some of the enumerated sections
\definecolor{manchet}{RGB}{75,75,75}
\newenvironment{manchet}{\sffamily\color{manchet}\small}{\color{black}\normalsize}

% * Whitespace: Defines the space before an enumerated section
\newcommand{\whitespace}{}  % \phantom{.} \vspace{0cm}

% * arxiv: Creating a href in the references, to the arXiv page in question
\newcommand{\bibArxiv}[1]{\href{https://arxiv.org/abs/#1}{\texttt{[arXiv:#1]}}}

% * inspire: Creating a href in the references, to the inSPIREhep page in question
\newcommand{\bibInspire}[1]{\href{https://inspirehep.net/record/#1}{\texttt{[inSPIRE:#1]}}}

% * inspire: Creating a href in the references, to the CERN CDS page in question
\newcommand{\bibCDS}[1]{\href{https://cds.cern.ch/record/#1}{\texttt{[CDS:#1]}}}


% * vol: Making journal volume upright and bold-face 
\newcommand{\bibVol}[1]{{\upshape\bfseries#1}}

% * title: Typesetting author(s) in small caps
\newcommand{\bibAuthor}[1]{{\scshape#1}}

% * cleartoleftpage: Makes sure that the following is put out to an odd-numbered (left) page.
\newcommand*\cleartoleftpage{%
  \clearpage
  \ifodd\value{page}\hbox{}\newpage\fi
}


% * Declaring inverse hyperbolic trigonometric functions as math operators (omitted by standard)
\DeclareMathOperator{\arccosh}{arccosh}
\DeclareMathOperator{\arcsinh}{arcsinh}
\DeclareMathOperator{\arctanh}{arctanh}
\DeclareMathOperator*{\median}{median}

\newcommand{\MET}{E^{\mathrm{miss}}_{T}}

\newcommand{\Emiss}{E^{\mathrm{miss}}}

\newcommand{\vek}[1]{\boldsymbol{\mathbf{#1}}}

\newcommand{\Tr}{\,\text{Tr}}

\newcommand{\diag}{\,\text{diag}}

\newcommand{\lb}{\langle}

\newcommand{\rb}{\rangle}

\newcommand{\norm}[1]{|\hspace{-0.15em}| #1 |\hspace{-0.15em}|}

\newcommand{\ex}[1]{\left\langle #1 \right\rangle}

\newcommand{\rightarrowsub}[1]{ \xrightarrow[ ~ #1 ~ ]{} }
\newcommand{\rightarrowsup}[1]{ \xrightarrow[]{ ~ #1 ~ } }

\newcommand{\bra}[1]{\langle#1|}
\newcommand{\ket}[1]{|#1\rangle}
\newcommand{\bracket}[2]{\langle#1\,|\,#2\rangle}

\newcommand{\diff}[2]{\frac{\mathrm{d}#1}{\mathrm{d}#2}}
\newcommand{\ddiff}[2]{\frac{\mathrm{d}^{2}#1}{\mathrm{d}#2^{2}}}

\let\oldcancelto\cancelto
\renewcommand{\cancelto}[2]{\oldcancelto{\scriptstyle #1}{#2}}

% * Name macros.
\newcommand{\pilemc}{\textsc{pilemc}}
\newcommand{\pythia}{\textsc{pythia}}
\newcommand{\herwig}{\textsc{Herwig}}
\newcommand{\herwigpp}{{\herwig}++}
\newcommand{\rivet}{\textsc{Rivet}}
\newcommand{\gsl}{\textsc{gsl}}
\newcommand{\newwave}{\textsc{NewWave}}
\newcommand{\fastjet}{\textsc{FastJet}}
\newcommand{\Root}{\textsc{root}}
\newcommand{\madgraph}{\textsc{MadGraph}}
\newcommand{\geant}{\textsc{Geant}}
\newcommand{\jetrec}{\textsc{jetrec}}
\newcommand{\powheg}{\textsc{powheg}}
\newcommand{\sherpa}{\textsc{sherpa}}
