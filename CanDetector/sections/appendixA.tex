\onecolumn
\section{Resolution scans for Am and Fe}
\label{app:resolution-scans}
\begin{figure}[htb]
  \centering
  \foreach \n [count=\i] in {%
    am_100_1136,
    am_100_1191,
    am_100_1244,
    am_100_1297,
    am_100_1351,
    am_100_1399,
    am_40_1397,
    am_40_1455}{
   \begin{subfigure}{.42\linewidth}
        
         \includegraphics[width=\linewidth]{graphics/\n}
        \caption{\detokenize\expandafter{\n}}
      \end{subfigure}
    }
  \end{figure}
  \begin{figure}[htb]\ContinuedFloat
  \centering
  \foreach \n [count=\i] in {%
    am_40_1502,
    am_40_1559,
    am_20_1559,
    am_20_1603,
    am_20_1666,
    am_10_1665,
    am_10_1712,
    am_10_1758}{
   \begin{subfigure}{.42\linewidth}
        
         \includegraphics[width=\linewidth]{graphics/\n}
        \caption{\detokenize\expandafter{\n}}
      \end{subfigure}
    }
\end{figure}
  \begin{figure}[htb]\ContinuedFloat
  \centering
  \foreach \n [count=\i] in {%
    am_4_1757,
    am_4_1808,
    am_4_1858,
    am_4_1899,
    am_2_1900,
    am_2_1951,
    am_2_2001} {
   \begin{subfigure}{.42\linewidth}
        
         \includegraphics[width=\linewidth]{graphics/\n}
        \caption{\detokenize\expandafter{\n}}
      \end{subfigure}
    }
    \caption{Scan for different gains and voltages for Americium.}
    \label{fig:scan:americium}
\end{figure}

\begin{figure}[htb]
  \centering
  \foreach \n [count=\i] in {%
fe_100_1420,
fe_100_1470,
fe_100_1523,
fe_100_1572,
fe_100_1617,
fe_100_1717,
fe_40_1717,
fe_40_1801}{
   \begin{subfigure}{.42\linewidth}
        
         \includegraphics[width=\linewidth]{graphics/\n}
        \caption{\detokenize\expandafter{\n}}
      \end{subfigure}
    }
  \end{figure}
  \begin{figure}[htb]\ContinuedFloat
  \centering
  \foreach \n [count=\i] in {%
fe_40_1853,
fe_40_1901,
fe_20_1901,
fe_20_1978,
fe_10_1978,
fe_10_2000,
fe_10_2042,
fe_10_2080}{
   \begin{subfigure}{.42\linewidth}
        
         \includegraphics[width=\linewidth]{graphics/\n}
        \caption{\detokenize\expandafter{\n}}
      \end{subfigure}
    }
\end{figure}
  \begin{figure}[htb]\ContinuedFloat
  \centering
  \foreach \n [count=\i] in {%
fe_4_2124,
fe_4_2201,
fe_2_2201,
fe_2_2254,
fe_2_2303}{
   \begin{subfigure}{.42\linewidth}
        
         \includegraphics[width=\linewidth]{graphics/\n}
        \caption{\detokenize\expandafter{\n}}
      \end{subfigure}
    }
    \caption{Scan for different gains and voltages for Iron.}
    \label{fig:scan:iron}
  \end{figure}

\FloatBarrier
