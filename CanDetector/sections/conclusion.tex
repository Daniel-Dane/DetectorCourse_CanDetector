\section{Conclusion}
The laboratory work conducted here can be concluded by a succesful build and
operation of a proportional gas detector based on a cider can. A different setup
with a copper tube has also been constructed but was not operational due to a
electrical short between the andode wire and the outer part of the HV connector.
Supplied data was instead analysed in this report for copper and aluminium
tubes.

The amplifying circuit used to measure the voltage from the detector has been
succesfully calibrated and found a gain of the preamplifier of
$G_\mathrm{pre}=\SI{6.41 \pm 0.01}{}$ and the uncertainty on the coarse gain was
determined to be $\sigma G_\mathrm{coarse} = 0.1$.

Herafter a resolution scan has been performed varying both voltage and amplifier
gain to determine the optimal setting for the two samples of interest: Fe$^{55}$
and Am$^{241}$. Using a \SI{10}{\percent} uncertainty for the selection of
region of interest for performing fits yielded good results at \SI{1937}{\volt}
and a gain of $4$ for both samples. Additionally the charge collected by the
cider can detector has been compared to theoretical expecation under the
assumption that the detecotr is operating in the linear regime and it was found
an agreement within the $1\sigma$ taking into account uncertainties in
determining the detector dimensions, atmospheric pressure and voltage fluctuations.

Data collected for the two samples at optimal resolution as well as background samples have
been analysed for all detectors. Here the calibration was applied, background
subtracted using a bspline parametrization of the data and escape peaks were fitted using
Gaussians. Peaks in the spectra were succesfully identified, noted and compared
to their known values showing a near perfect match for the cider can and a good
agreement for the aluminium detector.