\section{Introduction}  \label{sec:Introduction}
After observing particle interactions and correspondingly their tracks with
bubble or streamer chambers strong increases in accuracy were required in order
to resolve reliably interaction vertices \cite{charpak_high-resolution_1984}.
The development of detectors with electronic readout and consequently the
possibility of stacking and locating them at known positions would therefore
prove to be a working and robust solution to this problem. Strong advantages of
the gas-flow proportional counters are known already since the 1950s \cite{hendee_gasflow_1956} and include
apart from the electronic readout an operation at atmospheric pressure and
therefore decreases the material budget between the interaction point and the
gaseous volume. As even low amounts of signal electrons are amplified in an
avalanche that depends, among others, on the voltage applied the gaseous
proportional counters are an effective device that can be configured towards
various application fields. It is therefore still in wide use
\cite{Mindur:2017nqn} and
\todo[inline]{insert here picture of stacked detectors / tubes to show how they
  can be useful}
