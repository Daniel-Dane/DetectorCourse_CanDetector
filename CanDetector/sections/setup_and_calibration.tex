\section{Experimental setup Calibration}

\subsection{Set-up}

The can was attached to the $\mathrm{P}_{10}$ gas supply with a flow of
approximately \SI{30}{\milli\liter\per\minute} for a couple of hours to remove
most of the air inside for an initial test and measurement. Before moving on to
the actual measurements, the detector was connected to the HV power supply and
to the preamplifier. The output signal goes from the preamplifier to the
oscilloscope through a spec. amplifier as well as from the preamplifier to the
multichannel analyser (MCA) connected to a laptop, which allows data collection
under the experiment. The setup was tested by feeding a few pulses with a pulse
generator through the detector. During the calibration a large noise was
observed due to radiations. To reduce this, an aluminium foil was used to cover
the endcap and effectively shield from radiations.

The can detector was shortly tested with a $^{55}\mathrm{Fe}$ source to check
whether a signal was visible and the test turned out to be successful. Before
carrying out the actual experiment, the can detector was flushed with
$\mathrm{P}_{10}$ gas overnight.



