\section{Analysis of systematics}
In the analysis of data from experiments, care must be taken to properly take into account systematic uncertainties in our measurements. In the case of the cider can detector, environmental and geometric factors can influence the overall performances of the drift chamber, namely the average number of carriers produced and collected at the electrodes. This number, called the \textit{gain} or \textit{multiplication factor}, can be approximated by the following formula, assuming that our detector is operating in the linear regime(\cite{gas_detect}):

\begin{equation}
  \label{eq:lnm}
  \ln(M)=\frac{\ln(2)}{\ln(r_{c}/r_{a})}\cdot\frac{V}{\Delta V}\cdot\ln\left[ \frac{V\rho_{o}}{ra\ln(r_{c}/r_{a})E_{min}(\rho_{o})\rho}\right]
\end{equation}

The meaning and values associated with each parameter of this equation is listed in table \ref{Tab:params}. One can see that most of these values either come from tabulated properties of the gas used, or are derived from the initial measurement of the beer can dimensions, listed in table \ref{Tab:cidercan_sizes} for reference. 

\begin{table}[htb]
	\begin{tabularx}{\linewidth}{X|X|p{2cm}}
			\textbf{Element}         & \textbf{Measurements}                                 & \textbf{Size}       \\ \hline
			Cider can diameter       &                                                       &                     \\
			Cider can wall thickness & $105 \pm 10 \ \mu$m                                   & $105 \pm 10 \ \mu$m \\
			Plastic tube diameter    & $5.89$ mm, $5.95$ mm, $6.00$ mm, $6.01$ mm, $6.01$ mm &                     \\
			Brass tube diameter      & $1.0$ mm                                              &                     \\
			Nylon screw diameter     & $7.8$ mm, $7.75$ mm                                   &                     \\
			HV connector diameter    & $9.37$ mm                                             &                     \\
			Anode wire diameter      & $50 \pm 10 \ \mu$m                                    & $50 \pm 10 \ \mu$m 
	\end{tabularx}
\caption{Measurements of the cider can experiment setup components.}
\label{Tab:cidercan_sizes}
\end{table}

The relationship between the measured quantities and the anode radius, is simply $r_{a} = d_{a}/2.$.



The measures of the different elements are noted in \ref{Tab:cidercan_sizes}


\begin{table}[]
	\begin{tabularx}{\linewidth}{X|X|p{2cm}}
		\textbf{Element} & \textbf{Measurements} {[}mm{]}                                                  & \textbf{Size} \\ \hline
		Copper tube, length                        & $149.06$, $149.10$, $149.06$, $149.09$                                 &      \\
		Copper tube, wall thickness                & $1.04$, $1.00$, $1.02$, $1.02$, $1.02$, $1.14$, $1.05$, $1.04$, $1.04$ &      \\
		Copper tube, inner radius                  & $19.97$, $19.98$, $19.99$, $19.88$, $19.70$, $19.80$, $19.96$, $20.00$ &      \\
		Copper tube, radiation hole                & $5.03$, $5.04$, $5.08$                                                 &      \\
		Teflon HV front end,  total length         & $36.82$, $36.76$                                                       &      \\
		Teflon HV front end, top part length       & $15.91$, $15.94$, $15.86$                                              &      \\
		Teflon HV front end, tube extremity radius & $19.84$, $19.85$, $19.86$                                              &      \\
		Teflon back end, total length              & $15.04$, $14.97$, $14.89$, $14.93$, $14.95$, $15.06$                   &      \\
		Teflon back end, top part length           & $5.05$, $4.98$, $4.95$                                                 &      \\
		Teflon back end, tube extremity radius     & $19.88$, $19.86$                                                       &      \\
		Brass tube diameter                        & $1.0$                                                                  &      \\
		Anode wire thickened                       & $0.025$                                                                &     
	\end{tabularx}
\caption{Measurements of the copper pipe experiment setup components.}
\label{Tab:coppercan_sizes}
\end{table}
