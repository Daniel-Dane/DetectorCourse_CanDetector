\section{Analysis of systematics}
\label{sec:systematics}
In the analysis of data from experiments, care must be taken to properly take into account systematic uncertainties in our measurements. In the case of the cider can detector, environmental and geometric factors can influence the overall performances of the drift chamber, namely the average number of carriers produced and collected at the electrodes. This number, called the \textit{gain} or \textit{multiplication factor}, can be approximated by the following formula, assuming that our detector is operating in the linear regime(\cite{gas_detect}):

\begin{adjustwidth}{-2cm}{}
\begin{align}
\ln(M)=\frac{\ln(2)}{\ln(r_{c}/r_{a})}\cdot\frac{V}{\Delta V}\cdot\ln\left[ \frac{V\rho_{o}}{ra\ln(r_{c}/r_{a})E_{min}(\rho_{o})\rho}\right]
\label{eq:lnm}
\end{align}
\end{adjustwidth}

The meaning and values associated with each parameter of this equation is listed in table \ref{Tab:params}. One can see that most of these values either come from tabulated properties of the gas used, or are derived from the initial measurement of the beer can dimensions, listed in table \ref{Tab:cidercan_sizes} for reference.

\begin{table*}[htb]
  \begin{tabularx}{\linewidth}{p{1.5cm}p{8cm}rl}
    \textbf{Variable}     & \textbf{Definition}                                                         & \textbf{Value}     & \textbf{Source}  \\
    \hline
    $r_{c}$                 & radius of the cathode                                                       & $3.121 \pm 0.003$      & Eq. \ref{eq:rcra}   \\
    &&&\\
    $r_{a}$                 & radius of the anode                                                         & $\SI{25 +- 2}{\micro\meter}$ & Eq. \ref{eq:rcra}   \\
    &&&\\
    $V$                    & operating voltage                                                           & 1-3 kV             & N/A                \\
    &&&\\
    $\Delta V$             & potential required to produce an additional electron                & $23.6 \pm 5.4$ V   &\cite{gas_detect}   \\
    &&&\\
    $E_{min}(\rho_{o})$      & \begin{tabular}[c]{@{}l@{}}Minimal electric field needed for ionisation\\(at standard pressure)\end{tabular}         & $48. \pm 3$ kV/cm  &\cite{gas_detect}   \\
    &&&\\
    $\rho_{o}/\rho$ & \begin{tabular}[c]{@{}l@{}}Standard density of the gas\\(compared to density at  T=273K and P = 1 bar)\end{tabular}  &                    &Eq. \ref{eq:gaslaw}, \cite{meteo}\\
    \hline
  \end{tabularx}
  \caption{List of the main systematics sources}
  \label{Tab:params}
\end{table*}

For the geometric properties, the relationship between the measured quantities and the anode radius, is simply $r_{a} = d_{a}/2.$. Meanwhile, the radii of the cathode is related to the other measurements via equation \ref{eq:rcra}.

\begin{align}
  \label{eq:rcra}
  r_{c} = \frac{D_{outer}}{2}-\tau
\end{align}

To determine the gas properties at the environmental conditions of the laboratory, one needed to measure the temperature and pressure of the room. With these values on hand, the ratio of gas densities can be determined by using the ideal gas law.

\begin{align}
  \label{eq:gaslaw}
  \frac{\rho}{\rho_{o}} = \frac{P}{P_{o}}\cdot \frac{T_{o}}{T}
\end{align}

During the measurements, the temperature stayed mostly constant, but the pressure varied over the course of the day, as shown in figure \ref{fig:pressure}. The uncertainty on the pressure was thus selected to be the largest pressure change with respect to standard pressure, over the course of a day.

\begin{figure}[htb]
  \includegraphics[width=0.5\textwidth]{graphics/pressure_monitoring.png}
  \caption{Atmospheric pressure in Helsinki during the spectra measurement. Source: \cite{meteo}}
  \label{fig:pressure}
\end{figure}

Given the parameters and uncertainties quoted in table \ref{Tab:params}, a MC simulation was created in order to compute the theoretical uncertainties on the expected multiplication factor as a function of operating voltage. In this systematics treatment, each parameter was drawn out of Gaussian shaped probability distribution, with a mean centred at a parameter's value and the width set as the quoted uncertainty. Figure \ref{final_lnm} shows the 1$\sigma$ and 2$\sigma$ confidence interval of the multiplication factor, after propagation of systematic uncertainties.

The theoretical expectation for the drift chamber's electron yield can be compared to the data obtained during the iron spectrum scan described in section \ref{sec:resolution_scans}. Given a number of MCA counts $Q_{mca}$, the charge accumulated at the electrodes of the drift chamber can be obtained with equation \ref{eq:M_exp}.

\begin{align}
  \label{eq:M_exp}
  Q_{detector} = \frac{Q_{MCA}}{G_{pre,mca}\cdot{G_{coarse}}}
\end{align}

where $G_{pre,mca}$ is the preamplifier gain in units of $d.c./V$ (as plotted in figure \ref{fig:preamp_gain_mca}),and $G_{coarse}$ is the coarse gain that was described and measured in section \ref{sec:coarse}. From that point, the multiplication factor of the experimental data is given by Eq. \ref{eq:Mexp}.

\begin{align}
  \label{eq:Mexp}
  M_{experimental} &= \frac{N_{carriers,out}}{N_{carriers per avalanche}} \nonumber \\
                   &= \frac{Q_{detector}}{n_{Fe^{5},Am^{241}}\cdot e}
\end{align}

Where $n_{Fe^{5},Am^{241}}$ is the average number of electrons-ion pairs produced by either iron-55 (227) or americium-241 (2290), as taken from \cite{can_paper}. The measured multiplication factors obtained in each voltage scan is shown in figure \ref{final_lnm}, along with the theoretical expectation for their values given the geometry of the can detector and the environmental conditions during the measurement. As one can see, the data collected is matches remarkably well the prediction, which has its $1\sigma$ and $2\sigma$ contours plotted in blue.

\begin{figure}[htb]
  \includegraphics[width=0.5\textwidth]{graphics/lnM_final_plot.pdf}
  \caption{Natural logarithm of the drift chamber's multiplication factor M. Red and green data points represent the results obtained with $Fe^{5}$ and $Am^{241}$ respectively. Blue bands show the theoretical expectation for the detector, given systematics uncertainties in the cider can geometry and gas properties.}
  \label{final_lnm}
\end{figure}

